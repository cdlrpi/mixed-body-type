%        File: dca-equaitons.tex
%     Created: Tue Jul 08 04:00 PM 2014 E
% Last Change: Tue Jul 08 04:00 PM 2014 E
%
\documentclass[a4paper]{article}

\usepackage{amssymb,amsmath}
\usepackage{mathtools}
\usepackage{isomath}
\usepackage{natbib}

\begin{document}
\section*{Equations of Motion for A GEBF Element}
The equations of motion for a GEBF element are  
\begin{align}
  \dot{\vectorsym{h}} &= \vectorsym{N}^{'} + \vectorsym{f}  
  \label{eq:eom_linear_momentum}
  \shortintertext{and} 
  \dot{\vectorsym{g}} &= -\matrixsym{\tilde{\dot{u}}} \vectorsym{h} + \left( \matrixsym{\tilde{x_{0}^{'}}} + \matrixsym{\tilde{u^{'}}} \right) \vectorsym{N} + \vectorsym{M}^{'} + \vectorsym{\tau} ,
  \label{eq:eom_angular_momentum}
\end{align}
as given by Bauchau \cite{bauchau2010flexible}. 

\section*{Formulation of the ``Two-Handle Equations'' of Motion}
Equations~(\ref{eq:eom_linear_momentum}~\&~\ref{eq:eom_angular_momentum}) can be expressed in matrix form as  
\begin{align}
  \begin{Bmatrix}
   \dot{\vectorsym{h}} \\ 
   \dot{\vectorsym{g}}
  \end{Bmatrix} &=
  \begin{bmatrix}
    \matrixsym{0} & \matrixsym{0} \\
    -\matrixsym{\tilde{\dot{u}}} & \matrixsym{0} 
  \end{bmatrix}
  \begin{Bmatrix}
   \vectorsym{h} \\ 
   \vectorsym{g}
  \end{Bmatrix} +
  \begin{bmatrix}
    \matrixsym{0} & \matrixsym{0} \\
    \matrixsym{\tilde{x_{0}^{'}}} + \matrixsym{\tilde{u^{'}}} & \matrixsym{0}
  \end{bmatrix}
  \begin{Bmatrix}
   \vectorsym{N} \\ 
   \vectorsym{M}
  \end{Bmatrix} +
  \begin{Bmatrix}
    \vectorsym{N^{'}} \\ 
    \vectorsym{M^{'}}
  \end{Bmatrix} +
  \begin{Bmatrix}
    \vectorsym{f} \\ 
    \vectorsym{\tau}
  \end{Bmatrix} \ , 
  \label{eq:eom_momentum}
\end{align} 
where the $\tilde{\left( \ \right)}$ is the skew symetric matrix of the vector quantity $\left( \  \right)$ and $\left( \ \right)^{'}$ is the derivative with respect to the coordinate along the beam axis of $\left( \ \right)$.

The translational and angular momentum of the GEBF element is 
\begin{align}
  \begin{Bmatrix}
   \vectorsym{h} \\ 
   \vectorsym{g}
  \end{Bmatrix} &=
  \matrixsym{M}
  \begin{Bmatrix}
   \dot{\vectorsym{u}} \\ 
   \dot{\vectorsym{\omega}}
  \end{Bmatrix} \ ,
  \label{eq:element_momentum}
  \shortintertext{which leads to}
  \begin{Bmatrix}
   \dot{\vectorsym{h}} \\ 
   \dot{\vectorsym{g}}
  \end{Bmatrix} &= 
  \matrixsym{M}
  \begin{Bmatrix}
   \vectorsym{a} \\ 
   \vectorsym{\alpha}
  \end{Bmatrix} \ .
  \label{eq:dmomentum_dt}
\end{align}
The equations of motion for the GEBF element can then be written as
\begin{align}
  \matrixsym{M}
  \begin{Bmatrix}
   \vectorsym{a} \\ 
   \vectorsym{\alpha}
  \end{Bmatrix} &=
  \begin{bmatrix}
    \matrixsym{0} & \matrixsym{0} \\
    -\matrixsym{\tilde{\dot{u}}} & \matrixsym{0} 
  \end{bmatrix}
  \begin{Bmatrix}
   \vectorsym{h} \\ 
   \vectorsym{g}
  \end{Bmatrix} +
  \begin{bmatrix}
    \matrixsym{0} & \matrixsym{0} \\
    \matrixsym{\tilde{x_{0}^{'}}} + \matrixsym{\tilde{u^{'}}} & \matrixsym{0}
  \end{bmatrix}
  \begin{Bmatrix}
   \vectorsym{N} \\ 
   \vectorsym{M}
  \end{Bmatrix} +
  \begin{Bmatrix}
    \vectorsym{N^{'}} \\ 
    \vectorsym{M^{'}}
  \end{Bmatrix} +
  \begin{Bmatrix}
    \vectorsym{f} \\ 
    \vectorsym{\tau}
  \end{Bmatrix} \ .
  \label{eq:eom_accel}
\end{align} 
The RHS of the above equation contains all of the applied and body forces on the GEBF element, therefore the above equation can be simplified in spatial notation as
\begin{align}
  \matrixsym{M} A_{cm}^{e} =& \ F_{ba} \ .
  \label{eq:eom_element_spatial}
\end{align}

Since these elements are to be used in a multibody system, it is more desirable to formulate the equations of motion for the handles of the element and include the constraint forces acting at the boundary nodes of the element.  The constraint forces on the boundary nodes can be included in Eq.~\eqref{eq:eom_element_spatial} resulting in 
\begin{align}
  \matrixsym{M} A_{cm}^{e} =& \ \matrixsym{S}^{10} F_{1c} + \matrixsym{S}^{20} F_{2c} + F_{ba} \ ,
  \label{eq:eom_element_cnstr}
\end{align}
where $\matrixsym{S}^{i0}$ is the shift matrix that when multiplied by the constraint force acting at handle $i$ produces and equivalent force  and an associated moment acting and the center of mass (reference point $0$). The acceleration of the of the element is then 
\begin{align}
  A_{cm}^{e} =& \ \matrixsym{M}^{-1} \left( \matrixsym{S}^{10} F_{1c} + \matrixsym{S}^{20} F_{2c} + F_{ba} \right) \ .
  \label{eq:accel_element}
\end{align}
For a body having two handles, the acceleration of handles $1$ and $2$ can be generated from Eq.~\eqref{eq:accel_element} using similar shift matrices yielding 
\begin{align}
  A_{1}^{e} =& \ \matrixsym{S}^{01} \matrixsym{M}^{-1} \matrixsym{S}^{10} F_{1c} + \matrixsym{S}^{01} \matrixsym{M}^{-1} \matrixsym{S}^{20} F_{2c} + \matrixsym{S}^{01} \matrixsym{M}^{-1} F_{ba}
  \shortintertext{and}
  A_{2}^{e} =& \ \matrixsym{S}^{02} \matrixsym{M}^{-1} \matrixsym{S}^{10} F_{1c} + \matrixsym{S}^{02} \matrixsym{M}^{-1} \matrixsym{S}^{20} F_{2c} + \matrixsym{S}^{02} \matrixsym{M}^{-1} F_{ba}\ .
  \label{eq:two_handle}
\end{align}
The above equations can be simplified algebraically resulting in 
\begin{align}
  A_{1}^{e} =& \ \zeta_{11} F_{1c} + \zeta_{12} F_{2c} + \zeta_{13} F_{ba}
  \shortintertext{and}
  A_{2}^{e} =& \ \zeta_{21} F_{1c} + \zeta_{22} F_{2c} + \zeta_{23} F_{ba}\ ,
  \label{eq:two_handle_zeta}
\end{align}
which are now ready for use in the current DCA assembly and disassembly framework.  


\section{References}
\bibliographystyle{plain}
\bibliography{library}

\end{document}


