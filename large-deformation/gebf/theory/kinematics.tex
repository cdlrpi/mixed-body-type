%        File: kinematics.tex
%     Created: Tue Jul 08 04:00 PM 2014 E
% Last Change: Tue Jul 08 04:00 PM 2014 E

\documentclass[a4paper]{article}
\usepackage{amsmath}
\usepackage{accents}
\usepackage{mathtools}
\usepackage{isomath}

\newcommand{\ubar}[1]{\underaccent{\bar}{#1}}

\begin{document}
\section*{Geometrically Exact Beams Undergoing Large Deformations and Rotations}
\subsection*{Kinematics of A Generic Point}

%The large-deformation of a flexible beam may be decomposed into the deformation of the neutral axis of the beam and the rotation of the cross-section of the beam at any point along the neutral axis.  The position of any point ($\mathbf{r}^{P}$) on a flexible beam may be described by locating the centroid of a rigid-cross section $\left( \mathbf{r}_{0} \right)$ in the undeformed configuration, the deformation vector of the centroid $\left( \mathbf{\Delta} \right)$, and the location of the point in the cross-section $\left( \mathbf{s} \right)$.  The resulting vector is expressed in the Newtonian frame as    
\begin{align}
  \left\{ \mathbf{r}^{Q} \right\}_{N} =& \ \left\{ \mathbf{r}^{OP} \right\}_{N} 
  +  \underbrace{\left\{ \mathbf{r}^{PP'} \right\}_{N}}_{\mathbf{\Delta}}
  +  \prescript{N}{}{\ubar{\ubar{C}}}^{B}
  \left\{ \mathbf{r}^{P'Q} \right\}_{B} \ 
  \end{align}

\begin{alignat}{2}
  \left\{ \prescript{N}{}{\mathbf{v}}^{Q} \right\}_{N} =& \ \frac{d}{dt}\left\{ \mathbf{r}^{Q} \right\}_{N} \\
  =& \underbrace{\frac{d}{dt}\left\{ \mathbf{r}^{OP} \right\}_{N}}_{\mathbf{0}}
   + \underbrace{\frac{d}{dt}\left\{ \mathbf{r}^{PP'} \right\}_{N}}_{\mathbf{\dot{\Delta}}}  
   &&+ \frac{d}{dt}\left( \prescript{N}{}{\ubar{\ubar{C}}}^{B} \right)\left\{ \mathbf{r}^{P'Q} \right\}_{B}  \nonumber\\
   & &&+ \underbrace{\prescript{N}{}{\ubar{\ubar{C}}}^{B} \frac{d}{dt}\left\{ \mathbf{r}^{P'Q} \right\}_{B}}_{\mathbf{0}} 
  \end{alignat}
\begin{align}
  =& \ \left\{ \dot{\mathbf{\Delta}} \right\}_{N} 
     + \dot{\prescript{N}{}{\ubar{\ubar{C}}}^{B}} \left\{ \mathbf{r}^{P'Q} \right\}_{B} \\
  =& \ \left\{ \dot{\mathbf{\Delta}} \right\}_{N} 
  + \dot{\prescript{N}{}{\ubar{\ubar{C}}}^{B}} \ \ubar{\ubar{I}} \ \left\{ \mathbf{r}^{P'Q} \right\}_{B}  \\
  =& \ \left\{ \dot{\mathbf{\Delta}} \right\}_{N} 
     +  \underbrace{\dot{\prescript{N}{}{\ubar{\ubar{C}}}^{B}} 
     \left( \prescript{N}{}{\ubar{\ubar{C}}}^{B} \right)^{T}}_{\prescript{N}{}{\tilde{\omega}}^{B}} 
	\underbrace{\prescript{N}{}{\ubar{\ubar{C}}}^{B} \left\{ \mathbf{r}^{P'Q} \right\}_{B}}_{ \left\{ \mathbf{r}^{P'Q} \right\}_{N}} \\
  \left\{ \prescript{N}{}{\mathbf{v}}^{Q} \right\}_{N} =& \ \left\{ \dot{\mathbf{\Delta}} \right\}_{N}  
  +  \prescript{N}{}{\tilde{\omega}}^{B} \left\{ \mathbf{r}^{P'Q} \right\}_{N}
  \end{align}
\begin{alignat}{2}
  \left\{ \prescript{N}{}{\mathbf{a}}^{Q} \right\}_{N} =& \ \frac{d}{dt}\left\{ \mathbf{v}^{Q} \right\}_{N} \\
  =& \frac{d}{dt} \left\{ \dot{\mathbf{\Delta}} \right\}_{N} 
    + \frac{d}{dt} \left( \prescript{N}{}{\tilde{\omega}}^{B} \right) \left\{ \mathbf{r}^{P'Q} \right\}_{N}
    + \prescript{N}{}{\tilde{\omega}}^{B} 
    \underbrace{\frac{d}{dt} \left\{ \mathbf{r}^{P'Q} \right\}_{N}}_{\prescript{N}{}{\tilde{\omega}}^{B} \left\{ \mathbf{r}^{P'Q} \right\}_{N} } \\
  \left\{ \prescript{N}{}{\mathbf{a}}^{Q} \right\}_{N} =& \left\{ \ddot{\mathbf{\Delta}} \right\}_{N} 
    + \prescript{N}{}{\tilde{\alpha}}^{B}  \left\{ \mathbf{r}^{P'Q} \right\}_{N}
    + \prescript{N}{}{\tilde{\omega}}^{B} \prescript{N}{}{\tilde{\omega}}^{B} \left\{ \mathbf{r}^{P'Q} \right\}_{N}
  \end{alignat}

\subsection*{Interpolation of Deformations and Rotations}
\begin{align}
  \mathbf{\Delta} \left( x \right)  =& \ubar{\ubar{H}}\left( x \right) \begin{Bmatrix} \mathbf{\Delta}_{1}\\ \mathbf{\Delta}_{2} \end{Bmatrix} \\
  \prescript{N}{}{\ubar{\ubar{C}}}^{B} \left( x \right)  =& \ubar{\ubar{H}}\left( x \right) \begin{bmatrix} \prescript{N}{}{\ubar{\ubar{C}}}^{1}\\ \prescript{N}{}{\ubar{\ubar{C}}}^{2} \end{bmatrix}
  \label{}
\end{align}

\subsection*{Developing the Equations of Motion}
\begin{align}
  \pi\int_{0}^{L}\int_{0}^{r}\int_{0}^{r}\prescript{N}{}{\mathbf{a}}^{Q} \cdot \prescript{N}{}{\mathbf{v}}^{Q}_{r} \,dr\,dr\,dx- Q_{e} \cdot \prescript{N}{}{\mathbf{v}}^{Q}_{r}
\end{align}


%If $x$ is the coordinate along the beam axis, then
% \begin{align}
%   \left\{ \vec{r}_{0} \right\}_{B}  &= 
%   \begin{Bmatrix} x\\0 \end{Bmatrix} 
%   \shortintertext{and}
%   \left\{ \vec{r}_{0} \right\}_{N} &= ^{N}\mathbf{R}^{B_{0}}\left\{ \vec{r}_{0} \right\}_{B} \ ,
%   \label{}
% \end{align}
% where $^{N}\mathbf{R}^{B_{0}}$ rotates the beam-basis of the reference configuration to the Newtonian basis.  Additionally 
% \begin{align}
%   \left\{ \vec{s} \right\}_{N} = ^{N}\mathbf{R}^{B}\left\{ \vec{s} \right\}_{B} \ ,
%   \label{}
% \end{align}
% where $^{N}\mathbf{R}^{B}$ rotates the beam-basis of the deformed configuration (cross-section after rotation from the starting configuration)  to the Newtonian basis.\\
% 
% The velocity of any point $\left( \vec{v}_{P} \right)$ can be determined by differentiating $\vec{r}^{P}$ as
% \begin{align}
%   \left\{ \vec{v}_{P} \right\}_{N} &= \frac{d}{dt} \left\{ \vec{r}^{P} \right\}_{N}  
%   = \underbrace{\frac{d}{dt} \left\{ \vec{r}_{0} \right\}_{N}}_{0} 
%   + \frac{d}{dt} \left\{ \vec{\Delta} \right\}_{N} 
%   + \frac{d}{dt} \left\{ ^{N}\mathbf{R}^{B} \right\} \left\{ \vec{s} \right\}_{B}\\
%   &= \left\{ \dot{ \vec{\Delta}}\right\}_{N}  + ^{N}\dot{\mathbf{R}}^{B}  \left\{ \vec{s} \right\}_{B}
%   \label{eq:velocity1}
%   \end{align}
% 
% If Eq.~\eqref{eq:velocity1} is then left multiplied by the identity and the identity is then replaced with $\ubar{\ubar{R}}\ubar{\ubar{R}}^{T}$, the skew-symmetric angular velocity matrix can be extracted from the above expression as follows:
% \begin{align}
%   \left\{ \vec{v} \right\}_{I} 
%   &=  \left\{ \dot{ \vec{u}}\right\}_{I}
%   + \ubar{\ubar{I}} \ubar{\ubar{\dot{R}}} \ubar{\ubar{R}}_{0} \left\{ \vec{s} \right\}_{I}  \\
%   \left\{ \vec{v} \right\}_{I} 
%   &=  \left\{ \dot{ \vec{u}}\right\}_{I} + \ubar{\ubar{R}} 
%   \underbrace{\ubar{\ubar{R}}^{T} \ubar{\ubar{\dot{R}}}}_{\left\{ \tilde{\omega} \right\}_{B}} \ubar{\ubar{R}}_{0} \left\{ \vec{s} \right\}_{I}  
% \end{align}
% This results in an expression of the velocity of any point as a function of the time rate of change of the deformation vector and the angular velocity of the beam cross section as   
% \begin{align}
%   \left\{ \vec{v} \right\}_{I}
%   &= \left\{ \dot{ \vec{u}}\right\}_{I}
%   + \ubar{\ubar{R}} \left\{ \tilde{\omega} \right\}_{B} \ubar{\ubar{R}}_{0}  \left\{ \vec{s} \right\}_{I},
%   \label{eq:velocitysimp}
% \end{align}
% where $\left\{ \tilde{\omega} \right\}_{B}$ denotes the skew-symetric angular velocity matrix expressed in the material basis.  Equation~\eqref{eq:velocitysimp} is different from the simplification given also in equation~(16.52) of \textit{Flexible Multibody Dynamics}, which is
% \begin{align}
%   \ubar{v} &= \ubar{\dot{u}} + \ubar{\ubar{R}} \ubar{\ubar{R}}_{0} \tilde{\omega}^{*}s^{*}.
%   \label{eq:16522}
% \end{align}
% 
% %An alternate derivation can yield a similar result to Eq.~\eqref{eq:16522}, but will not result in an expression matching Eq.~\eqref{eq:16521}.  This is accomplished by replacing the product of the rotation matrices associated with the transformation from the reference configuration to the material configuration $\left( \ubar{\ubar{R}} \right)$ and the transformation from the inertial frame to the reference configuration $\left( \ubar{\ubar{R}}_{0} \right)$ can be replaced by a single matrix $\left( \ubar{\ubar{R}}_{BI} = \ubar{\ubar{R}}\ubar{\ubar{R}}_{0} \right)$.  The time derivative of the position of any point on the deformable beam can then be written as
% \begin{align}
%   \frac{d}{dt}\left\{ \vec{X}\left( \alpha_{1}, \alpha_{2}, \alpha_{3} \right) \right\}_{I}  
%   &= \frac{d}{dt} \left\{ \vec{x}_{0} \right\}_{I} 
%   + \frac{d}{dt} \left\{ \vec{u} \right\}_{I} 
%   + \frac{d}{dt} \left\{ \ubar{\ubar{R}}_{BI} \right\}
%   \left( \left\{ \vec{w} \right\}_{I} + \alpha_{2} \hat{i}_{2} + \alpha_{3} \hat{i}_{3} \right), 
%   \label{eq:DdefDt2}
% \end{align}
% again neglecting the warp.  
% 
% With a similar simplification, the velocity of any point on the beam can be expressed as  
% \begin{align}
%   \left\{ \vec{v} \right\}_{I} 
%   &= \left\{ \dot{ \vec{u}}\right\}_{I} 
%   + \ubar{\ubar{\dot{R}}}_{BI} 
%   \left( \vec{0} + \alpha_{2} \hat{i}_{2} + \alpha_{3} \hat{i}_{3} \right) \ .
%   \label{eq:velocity}
% \end{align}
% This equation can be further simplified as follows: 
% \begin{align}
%   \left\{ \vec{v} \right\}_{I} 
%   &= \left\{ \dot{ \vec{u}}\right\}_{I} 
%   + \ubar{\ubar{\dot{R}}}_{BI} 
%   \underbrace{\left( \vec{0} + \alpha_{2} \hat{i}_{2} 
%   + \alpha_{3} \hat{i}_{3} \right)}_{\left\{ \vec{s} \right\}_{I}}, \\
%   \left\{ \vec{v} \right\}_{I} 
%   &=  \left\{ \dot{ \vec{u}}\right\}_{I}
%   + \ubar{\ubar{\dot{R}}}_{BI} \left\{ \vec{s} \right\}_{I}.
% \end{align}
% By left multiplying by the identity and replacing it with $\ubar{\ubar{R}}_{BI}\ubar{\ubar{R}}_{BI}^{T}$, the skew-symmetric angular velocity matrix can be extracted from the above expression as follows:
% \begin{align}
%   \left\{ \vec{v} \right\}_{I} 
%   &=  \left\{ \dot{ \vec{u}}\right\}_{I}
%   + \ubar{\ubar{I}} \ubar{\ubar{\dot{R}}}_{BI} \left\{ \vec{s} \right\}_{I}  \\
%   \left\{ \vec{v} \right\}_{I} 
%   &=  \left\{ \dot{ \vec{u}}\right\}_{I}
%   + \ubar{\ubar{R}}_{BI} \underbrace{\ubar{\ubar{R}}^{T}_{BI} \ubar{\ubar{\dot{R}}}_{BI}}_{\left\{ \tilde{\omega} \right\}_{B}} \left\{ \vec{s} \right\}_{I}  \\
%   \left\{ \vec{v} \right\}_{I} 
%   &= \left\{ \dot{ \vec{u}}\right\}_{I}
%   + \ubar{\ubar{R}}_{BI} \left\{ \tilde{\omega} \right\}_{B} \left\{ \vec{s} \right\}_{I}   
% \end{align}
% This results in an expression of the velocity of any point as a function of the time rate of change of the deformation vector and the angular velocity of the beam element as   
% \begin{align}
%   \left\{ \vec{v} \right\}_{I}
%   &= \left\{ \dot{ \vec{u}}\right\}_{I}
%   + \ubar{\ubar{R}} \ubar{\ubar{R}}_{0} \left\{ \tilde{\omega} \right\}_{B} \left\{ \vec{s} \right\}_{I}.
%   \label{eq:velocitysimp2}
% \end{align}
% 
% This seems to suggest that I am missing something, or perhaps my understanding of the problem description is incorrect.  For example, $\ubar{\ubar{\dot{R}}}_{BI}$ may not necissarily be equal to $\ubar{\ubar{\dot{R}}} \ubar{\ubar{R}}_{0}$, which may account for the descrepancy between Eq.~\eqref{eq:velocitysimp} and Eq.~\eqref{eq:16522}.  However, this would mean that my understanding of the reference configuration ($b_{i}$), which is that it is not moving, would be incorrect. \\ 
% \\
% Any advice would be greatly appreciated!  
\end{document}
