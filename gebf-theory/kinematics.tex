%        File: kinematics.tex
%     Created: Tue Jul 08 04:00 PM 2014 E
% Last Change: Tue Jul 08 04:00 PM 2014 E

\documentclass[a4paper]{article}
\usepackage{amsmath}
\usepackage{accents}
\usepackage{mathtools}
\usepackage{isomath}

\newcommand{\ubar}[1]{\underaccent{\bar}{#1}}

\begin{document}
\section*{Kinematics of a Highly-Flexible Beam}

From equation~(16.43) of \textit{Flexible Multibody Dynamics}, \begin{align}
  \vec{X}\left( \alpha_{1}, \alpha_{2}, \alpha_{3} \right)  
  &= \vec{X}_{0} + w_{1}\bar{B}_{1} + \left( w_{2} + \alpha_{2} \right)\bar{B}_{2}
  + \left( w_{3} + \alpha_{3} \right)\bar{B}_{3}
  \label{eq:1643}
\end{align}
the displacement of any point on the beam may be written as 
\begin{align}
  \left\{ \vec{X}\left( \alpha_{1}, \alpha_{2}, \alpha_{3} \right) \right\}_{I}  
  &= \left\{ \vec{x}_{0} \right\}_{I} 
  + \left\{ \vec{u} \right\}_{I} 
  + \left( \ubar{\ubar{R}} \ubar{\ubar{R}}_{0} \right) 
  \left( \left\{ \vec{w} \right\}_{I} + \alpha_{2} \hat{i}_{2} + \alpha_{3} \hat{i}_{3} \right),
  \label{eq:deformation}
\end{align}
which is given by equation~(16.44) of \textit{Flexible Multibody Dynamics}.  In Eq.~\eqref{eq:deformation}, the subscripts $B$ denotes vector quantities expressed in the material basis and the subscript $I$ denotes quantities expressed in the inertial basis.  To obtain the velocity of any point on the beam the time derivative of Eq~\eqref{eq:deformation} is taken, resulting in
\begin{align}
  \frac{d}{dt}\left\{ \vec{X}\left( \alpha_{1}, \alpha_{2}, \alpha_{3} \right) \right\}_{I}  
  &= \frac{d}{dt} \left\{ \vec{x}_{0} \right\}_{I} 
  + \frac{d}{dt} \left\{ \vec{u} \right\}_{I} 
  + \frac{d}{dt} \left\{ \ubar{\ubar{R}} \ubar{\ubar{R}}_{0} \right\}
  \left( \left\{ \vec{w} \right\}_{I} + \alpha_{2} \hat{i}_{2} + \alpha_{3} \hat{i}_{3} \right).
  \label{eq:DdefDt}
\end{align}
% if the contribution of the warp $\left( \vec{w} \right)$ is neglected.
%and because there is no change of the reference configuration with respect to the inertial frame $\left( \frac{d}{dt} \left\{ \ubar{\ubar{R}}_{0} \right\} = 0 \right)$.  

After simplification the velocity of any point on the beam can be expressed as  
\begin{align}
  \left\{ \vec{v} \right\}_{I} 
  &= \left\{ \dot{ \vec{u}}\right\}_{I} 
  + \ubar{\ubar{\dot{R}}}\ubar{\ubar{R}}_{0} 
  \left( \vec{0} + \alpha_{2} \hat{i}_{2} + \alpha_{3} \hat{i}_{3} \right) \ ,
\end{align}
if the contribution of the warp vector is neglected.  This equation can be further simplified as follows: 
\begin{align}
  \left\{ \vec{v} \right\}_{I}  
  &= \left\{ \dot{ \vec{u}}\right\}_{I} 
  +  \ubar{\ubar{\dot{R}}} \ubar{\ubar{R}}_{0} 
  \underbrace{\left( \vec{0} + \alpha_{2} \hat{i}_{2} 
  + \alpha_{3} \hat{i}_{3} \right)}_{\left\{ \vec{s} \right\}_{I}}, \\
  \left\{ \vec{v} \right\}_{I} 
  &=  \left\{ \dot{ \vec{u}}\right\}_{I} 
  +  \ubar{\ubar{\dot{R}}} \ubar{\ubar{R}}_{0} \left\{ \vec{s} \right\}_{I}.
  \label{eq:velocity1}
\end{align}

Equation~\eqref{eq:velocity1} agrees with equation~(16.52) of \textit{Flexible Multibody Dynamics}, which is
\begin{align}
  \ubar{v} = \ubar{\dot{u}} + \ubar{\ubar{\dot{R}}} \ubar{\ubar{R}}_{0} \ubar{s}^{*}
  \label{eq:16521}
\end{align}
If Eq.~\eqref{eq:velocity1} is then left multiplied by the identity and the identity is then replaced with $\ubar{\ubar{R}}\ubar{\ubar{R}}^{T}$, the skew-symmetric angular velocity matrix can be extracted from the above expression as follows:
\begin{align}
  \left\{ \vec{v} \right\}_{I} 
  &=  \left\{ \dot{ \vec{u}}\right\}_{I}
  + \ubar{\ubar{I}} \ubar{\ubar{\dot{R}}} \ubar{\ubar{R}}_{0} \left\{ \vec{s} \right\}_{I}  \\
  \left\{ \vec{v} \right\}_{I} 
  &=  \left\{ \dot{ \vec{u}}\right\}_{I} + \ubar{\ubar{R}} 
  \underbrace{\ubar{\ubar{R}}^{T} \ubar{\ubar{\dot{R}}}}_{\left\{ \tilde{\omega} \right\}_{B}} \ubar{\ubar{R}}_{0} \left\{ \vec{s} \right\}_{I}  
\end{align}
This results in an expression of the velocity of any point as a function of the time rate of change of the deformation vector and the angular velocity of the beam cross section as   
\begin{align}
  \left\{ \vec{v} \right\}_{I}
  &= \left\{ \dot{ \vec{u}}\right\}_{I}
  + \ubar{\ubar{R}} \left\{ \tilde{\omega} \right\}_{B} \ubar{\ubar{R}}_{0}  \left\{ \vec{s} \right\}_{I},
  \label{eq:velocitysimp}
\end{align}
where $\left\{ \tilde{\omega} \right\}_{B}$ denotes the skew-symetric angular velocity matrix expressed in the material basis.  Equation~\eqref{eq:velocitysimp} is different from the simplification given also in equation~(16.52) of \textit{Flexible Multibody Dynamics}, which is
\begin{align}
  \ubar{v} &= \ubar{\dot{u}} + \ubar{\ubar{R}} \ubar{\ubar{R}}_{0} \tilde{\omega}^{*}s^{*}.
  \label{eq:16522}
\end{align}

An alternate derivation can yield a similar result to Eq.~\eqref{eq:16522}, but will not result in an expression matching Eq.~\eqref{eq:16521}.  This is accomplished by replacing the product of the rotation matrices associated with the transformation from the reference configuration to the material configuration $\left( \ubar{\ubar{R}} \right)$ and the transformation from the inertial frame to the reference configuration $\left( \ubar{\ubar{R}}_{0} \right)$ can be replaced by a single matrix $\left( \ubar{\ubar{R}}_{BI} = \ubar{\ubar{R}}\ubar{\ubar{R}}_{0} \right)$.  The time derivative of the position of any point on the deformable beam can then be written as
\begin{align}
  \frac{d}{dt}\left\{ \vec{X}\left( \alpha_{1}, \alpha_{2}, \alpha_{3} \right) \right\}_{I}  
  &= \frac{d}{dt} \left\{ \vec{x}_{0} \right\}_{I} 
  + \frac{d}{dt} \left\{ \vec{u} \right\}_{I} 
  + \frac{d}{dt} \left\{ \ubar{\ubar{R}}_{BI} \right\}
  \left( \left\{ \vec{w} \right\}_{I} + \alpha_{2} \hat{i}_{2} + \alpha_{3} \hat{i}_{3} \right), 
  \label{eq:DdefDt2}
\end{align}
again neglecting the warp.  

With a similar simplification, the velocity of any point on the beam can be expressed as  
\begin{align}
  \left\{ \vec{v} \right\}_{I} 
  &= \left\{ \dot{ \vec{u}}\right\}_{I} 
  + \ubar{\ubar{\dot{R}}}_{BI} 
  \left( \vec{0} + \alpha_{2} \hat{i}_{2} + \alpha_{3} \hat{i}_{3} \right) \ .
  \label{eq:velocity}
\end{align}
This equation can be further simplified as follows: 
\begin{align}
  \left\{ \vec{v} \right\}_{I} 
  &= \left\{ \dot{ \vec{u}}\right\}_{I} 
  + \ubar{\ubar{\dot{R}}}_{BI} 
  \underbrace{\left( \vec{0} + \alpha_{2} \hat{i}_{2} 
  + \alpha_{3} \hat{i}_{3} \right)}_{\left\{ \vec{s} \right\}_{I}}, \\
  \left\{ \vec{v} \right\}_{I} 
  &=  \left\{ \dot{ \vec{u}}\right\}_{I}
  + \ubar{\ubar{\dot{R}}}_{BI} \left\{ \vec{s} \right\}_{I}.
\end{align}
By left multiplying by the identity and replacing it with $\ubar{\ubar{R}}_{BI}\ubar{\ubar{R}}_{BI}^{T}$, the skew-symmetric angular velocity matrix can be extracted from the above expression as follows:
\begin{align}
  \left\{ \vec{v} \right\}_{I} 
  &=  \left\{ \dot{ \vec{u}}\right\}_{I}
  + \ubar{\ubar{I}} \ubar{\ubar{\dot{R}}}_{BI} \left\{ \vec{s} \right\}_{I}  \\
  \left\{ \vec{v} \right\}_{I} 
  &=  \left\{ \dot{ \vec{u}}\right\}_{I}
  + \ubar{\ubar{R}}_{BI} \underbrace{\ubar{\ubar{R}}^{T}_{BI} \ubar{\ubar{\dot{R}}}_{BI}}_{\left\{ \tilde{\omega} \right\}_{B}} \left\{ \vec{s} \right\}_{I}  \\
  \left\{ \vec{v} \right\}_{I} 
  &= \left\{ \dot{ \vec{u}}\right\}_{I}
  + \ubar{\ubar{R}}_{BI} \left\{ \tilde{\omega} \right\}_{B} \left\{ \vec{s} \right\}_{I}   
\end{align}
This results in an expression of the velocity of any point as a function of the time rate of change of the deformation vector and the angular velocity of the beam element as   
\begin{align}
  \left\{ \vec{v} \right\}_{I}
  &= \left\{ \dot{ \vec{u}}\right\}_{I}
  + \ubar{\ubar{R}} \ubar{\ubar{R}}_{0} \left\{ \tilde{\omega} \right\}_{B} \left\{ \vec{s} \right\}_{I}.
  \label{eq:velocitysimp2}
\end{align}

This seems to suggest that I am missing something, or perhaps my understanding of the problem description is incorrect.  For example, $\ubar{\ubar{\dot{R}}}_{BI}$ may not necissarily be equal to $\ubar{\ubar{\dot{R}}} \ubar{\ubar{R}}_{0}$, which may account for the descrepancy between Eq.~\eqref{eq:velocitysimp} and Eq.~\eqref{eq:16522}.  However, this would mean that my understanding of the reference configuration ($b_{i}$), which is that it is not moving, would be incorrect. \\ 
\\
Any advice would be greatly appreciated!  
\end{document}
